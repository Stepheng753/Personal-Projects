\documentclass[12pt]{article}
\usepackage[margin = 1in]{geometry}
\usepackage{amsmath}
\usepackage{amssymb}
\usepackage{cancel}

\begin{document}

\begin{center}
	\textbf{Final Project - Complex Variable Analysis} \\
	\textbf{Due:\ Wednesday, December 18th, 2019} \\
	\textbf{Authors:}\\
	\textbf{Javier Anguiano}\\
	\textbf{Elvidio Hidalgo}\\
	\textbf{Emily Boyd}\\
	\textbf{Stephen Giang}\\
\end{center}

\medskip Find the function, which is analytic throughout circle C and it's interior, whose center is at the origin and whose radius is unity, and has the value:
\begin{align}
\frac{a - \cos (\theta)}{a^2 - 2a\cos(\theta) + 1} + i\frac{\sin (\theta)}{a^2 - 2a\cos(\theta) + 1}
\end{align}

\begin{center}
(note: $a > 1$ and $\theta$ is the vectorial angle)
\end{center}

at points on the circumference of C.

\begin{align}
f^{(n)} (0) 
&= \frac{n!}{2\pi i} \int_{C} \frac{f(z)}{z^{n+1}} dz
\\
&= \frac{n!}{2\pi i} \int_{0}^{2 \pi} e^{-n i \theta} \cdot i d \theta \cdot \frac{a - \cos( \theta ) + i \sin ( \theta)}{a^2 - 2a\cos(\theta) + 1}
\\
&\text{ let} z = e^{i \theta}
\\
&= \frac{n!}{2\pi i} \int_{0}^{2 \pi} \frac{e^{-n i \theta}}{a - e^{i \theta}} d \theta
\\
&= \frac{n!}{2\pi i} \int_{C} \frac{dz}{z^n (a-z)}
\\
&= \left.\frac{d^n}{dz^n} \frac{1}{a - z}\right|_{z = 0}
\\
&= \frac{n!}{a^{n+1}}
\end{align}

Therefore with MacLaurin's Theorem 
\begin{align}
f(z) 
&= f(0) + z f'(0) + \frac{z^2}{2!} f''(0) + ... + \frac{z^n}{n!} f^{(n)} (0).
\\
f(z) 
&= \sum_{n = 0}^{\inf} \frac{z^n}{a^{n+1}}
\\
&= (a - z)^{-1} 
\\
&\forall x \in C
\end{align}
\newpage


State why the components of velocity can be obtained from the stream function by means of the equations:
\begin{align}
& p(x,y) = \psi_y(x,y) 
& q(x,y) = -\psi_x(x,y)
\end{align}
Let the vector $\vec{V}(x,y) = p(x,y) + iq(x,y) $ represent the velocity of a particle of a fluid at a point $(x,y)$.
\\\\
Let $\phi (x,y) $ represents the velocity potential of a particle of fluid at a point $(x,y)$.
\begin{align}
\phi (x,y) = \int_{(x_0,y_0)}^{(x,y)} p(s,t) ds + \int_{(x_0,y_0)}^{(x,y)} q(s,t) dt
\end{align}
When we differentiate both sides with respect to $x$ and with respect to $y$, we get 
\begin{align}
\phi_x (x,y) &= p(x,y)
\\
\phi_y (x,y) &= q(x,y)
\end{align}
So $\vec{V} (x,y) = \phi _x (x,y) + i \phi_y (x,y)$, which makes $\vec{V}(x,y)$ the gradient of $\phi (x,y)$.
\\\\
Let $\psi (x,y)$ be a harmonic conjugate to $\phi$ (x,y). That is, Let $\psi (x,y)$ and $\phi(x,y)$ be harmonic such that they satisfy the Cauchy-Riemann Equations: 
\begin{align}
&\psi_x(x,y) = \phi_y(x,y) 
&\psi_y(x,y) = -\phi_x(x,y) 
\end{align}
The Complex Potential is: $F(z) = \phi(x,y) + i\psi(x,y) $
\begin{align}
F'(z) &= \phi_x (x,y) + i\psi_x(x,y)
\\
&= \phi_y (x,y) - i\psi_y(x,y)
\end{align}
$\vec{V}$(x,y) is the Complex Conjugate of F'(z):
\begin{align}
\vec{V}(x,y) &= \bar{F}'(z)
\\
&= \phi_x(x,y) + i \phi_y(x,y)
\\
&= -1(\psi_y(x,y) + i\psi_x(x,y))
\\
&= i^2
\end{align}


\end{document}